\documentclass[12pt]{article}
\usepackage{mla13}
\usepackage{mathptmx}
\firstname{Evan}
\lastname{Walter}
\professor{Ms. Nygren}
\class{Writ 1301}
\title{An Excursion to the Black Hills}

\begin{document}
\makeheader
\hspace{0.5in}
If I were given the choice to remember a week of my life down to the finest sensory details, I would, without a doubt, use it to remember the experiences I had on my family's trip to the Black Hills.  It's hard to place exactly what it was that made me love this trip so much, as I'd been on many other sightseeing trips before.  Maybe it was the combination of the awe-inspiring scenery I saw and the fact that we weren't scrambling to see everything we could, or maybe I just liked hanging out in a new place for a week.  Either way, looking back, I feel like I could go there a thousand times and never tire of the landscapes that felt so foreign to someone like me who'd lived in Minnesota for years.

While the vacation itself was immensely fun, the long ride which the vacation demanded of us left a lot to be desired.  As a family, we didn't really talk to each other in the car much, and often resorted to doing whatever it was that we usually did in the car, with my dad driving, my mom reading, my brother watching videos and myself listening to music.  When I listen to music on long car rides, I usually look out the window at passing scenery;  I'm always on the right side so my vision isn't obscured by cars passing in another lane (except on roads with four or more lanes, of course).  However, on this ride, the scenery left something to be desired.  There were about as many hills as there are cats at the north pole, and the expanse was about as much of a feast for the eyes as plain tofu is for the mouth.  60\% of the time, the sky would converge at the horizon with rows of crops.  For the most part, I wasn't bothered by it, as this scene was just a sort of moving backdrop behind my thoughts.  Listening to music and letting my mind wander is my war tactic against boredom when sitting through long car rides.

When we neared our destination, being just shy of twenty minutes away from it, we stopped at a place my dad had been wanting to try since planning this trip.  The food came much appreciated, as I'd been tired of subsisting off of junk food for the past few hours, though one could argue that the pizza was just as bad for me, if not worse.  The decor in the joint was rather dry.  The best way I could really describe such a place would be as the abandoned child of a bathroom, a pub, and a gymnasium, and I'll leave that as something for the imagination to chew on.  There was a steep disparity between the ambience and the quality of our food, though.  The pizza we got was warm, flavorful and had a marvelous composition.  It didn't leave 16 inches of thin cheese strings between your chin and the plate when you tried your hardest to take a bite from it.  The pepperonis came clean off likewise.  Had I not jumped the gun and subjected my mouth to Hell Junior by taking a bite far too soon, it would've been perfect.  Maybe that's what people refer to as user error, though.

Arriving at a plateau-like location, we hurriedly exited the car once we'd pulled into the driveway.  Unanimously, we stretched in order to ease the dull ache of sitting in a car for hours on end.  Within seconds of entering the foyer, I already felt at home--  decor on the walls and trinkets on shelves all worked together to create an inviting wooden aesthetic to match the scene I saw outdoors.  It all really appealed to not only the eyes, but also to the mind.  As for its layout, the house wasn't too large, but used its size and availability of a basement well.  In nearly every room, there was probably room to run without much danger of bashing your head and sending yourself to the hospital.  The decor throughout the house rivaled that of what I'd tasted in the foyer, each being themed differently.  You could find that ``peace, love, happiness,'' was plastered on a pillow in one bedroom, and another would be decorated with a borderline oppressive amount of nautical imagery.

The experiences I had on this trip were just as diverse as the rooms we stayed in.  It would be a sin to forget about one excursion in particular:  the Badlands.  What it exhibited was more akin to an awesome brilliance than beauty, with terrain you rarely see the likes of in your day-to-day life.  Our car weaved in and out of seemingly extraterrestrial mountains and masses of stone and clay.  I might as well have been on Mars, setting aside the fact that I would've been long since dead.  Eventually, we arrived at a place which presented land that was a lot more flat than what we had been driving through.  It was scattered with small man-made elements like bridges to cross narrow creeks, and paths for many chattering families that were just as excited as we were to hike in such a place.  I put on sunscreen-- reluctantly, as I really don't like the feeling it leaves on my skin-- and followed my family out to the hiking trails.  We weaved around smaller protruding masses than the monolithic ones we'd seen before.  The vegetation around us conversed with the wind, save for the cacti, which stood firm in their cracked tan soil.  While eavesdropping on their conversations, we noted the rat-tat-tat of a rattlesnake, who impolitely cut our hike short. 

Another interesting adventure we departed for on one day was a visit to the Needles Highway in Custer State Park, which was a really impressive place.  The first thing anybody would notice about such a place is that it's very similar to the Badlands in the extraterrestrial air it seems to exert.  Seemingly inexplicable stone peaks protrude from the ground in a fashion similar to stalagmites.  Man-made roads snaked all about, looping around and even cutting through the stones.  Hardy coniferous trees stood proud among the tall stones, almost as if they were trying to mimic them.  I'm not sure I can speak for my entire family, but I was taken aback by the grand display that the stone and flora put on.  I think that places like these really make you appreciate the concept of sunlight and light itself that much more, especially when you see the intricate little shadows cast onto rock faces by their irregularities and the highlights on trees.  The experience we had when hiking there was just as great.  The paths for hiking were much more well-carved than I would expect, given we were in terrain that appeared so unforgiving.  Picturesque lakes with a dark turquoise tint crowded some of the rocks, which sported trees that matched the water's color very well.  Some families were swimming in these lakes and chattering among themselves merrily, and though I didn't end up doing the same, it was a tantalizing idea.  We were as beaten as the figurative dead horse once we'd arrived back at our car, but satisfied all the same.


\end{document}
